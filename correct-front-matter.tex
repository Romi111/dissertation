\documentclass[]{msu-thesis}
\usepackage{lmodern}
\usepackage{amssymb,amsmath}
\usepackage{ifxetex,ifluatex}
\usepackage{fixltx2e} % provides \textsubscript
\ifnum 0\ifxetex 1\fi\ifluatex 1\fi=0 % if pdftex
  \usepackage[T1]{fontenc}
  \usepackage[utf8]{inputenc}
\else % if luatex or xelatex
  \ifxetex
    \usepackage{mathspec}
  \else
    \usepackage{fontspec}
  \fi
  \defaultfontfeatures{Ligatures=TeX,Scale=MatchLowercase}
\fi
% use upquote if available, for straight quotes in verbatim environments
\IfFileExists{upquote.sty}{\usepackage{upquote}}{}
% use microtype if available
\IfFileExists{microtype.sty}{%
\usepackage{microtype}
\UseMicrotypeSet[protrusion]{basicmath} % disable protrusion for tt fonts
}{}
\usepackage[margin=1in]{geometry}
\usepackage{hyperref}
\hypersetup{unicode=true,
            pdftitle={Examining Work With Data in STEM Education Through the Lens of Engagement Theory: A Person-Oriented Approach Using an Experience Sampling Method},
            pdfauthor={Joshua M. Rosenberg},
            pdfborder={0 0 0},
            breaklinks=true}
\urlstyle{same}  % don't use monospace font for urls
\usepackage{natbib}
\bibliographystyle{apalike}
\usepackage{longtable,booktabs}
\usepackage{graphicx,grffile}
\makeatletter
\def\maxwidth{\ifdim\Gin@nat@width>\linewidth\linewidth\else\Gin@nat@width\fi}
\def\maxheight{\ifdim\Gin@nat@height>\textheight\textheight\else\Gin@nat@height\fi}
\makeatother
% Scale images if necessary, so that they will not overflow the page
% margins by default, and it is still possible to overwrite the defaults
% using explicit options in \includegraphics[width, height, ...]{}
\setkeys{Gin}{width=\maxwidth,height=\maxheight,keepaspectratio}
\IfFileExists{parskip.sty}{%
\usepackage{parskip}
}{% else
\setlength{\parindent}{0pt}
\setlength{\parskip}{6pt plus 2pt minus 1pt}
}
\setlength{\emergencystretch}{3em}  % prevent overfull lines
\providecommand{\tightlist}{%
  \setlength{\itemsep}{0pt}\setlength{\parskip}{0pt}}
\setcounter{secnumdepth}{5}
% Redefines (sub)paragraphs to behave more like sections
\ifx\paragraph\undefined\else
\let\oldparagraph\paragraph
\renewcommand{\paragraph}[1]{\oldparagraph{#1}\mbox{}}
\fi
\ifx\subparagraph\undefined\else
\let\oldsubparagraph\subparagraph
\renewcommand{\subparagraph}[1]{\oldsubparagraph{#1}\mbox{}}
\fi

%%% Use protect on footnotes to avoid problems with footnotes in titles
\let\rmarkdownfootnote\footnote%
\def\footnote{\protect\rmarkdownfootnote}

%%% Change title format to be more compact
\usepackage{titling}

% Create subtitle command for use in maketitle
\newcommand{\subtitle}[1]{
  \posttitle{
    \begin{center}\large#1\end{center}
    }
}

%\setlength{\droptitle}{-2em}
%  \title{Examining Work With Data in STEM Education Through the Lens of Engagement Theory: A Person-Oriented Approach Using an Experience Sampling Method}
%  \pretitle{\vspace{\droptitle}\centering\huge}
%  \posttitle{\par}
%  \author{Joshua M. Rosenberg}
%  \preauthor{\centering\large\emph}
%  \postauthor{\par}
%  \predate{\centering\large\emph}
%  \postdate{\par}
%  \date{2017-11-24}
%

\frontmatter
%\pagenumbering{Roman}
\newpage

\newpage

\pagebreak

\pagebreak


\usepackage{booktabs}
\usepackage{amsthm}
\makeatletter
\def\thm@space@setup{%
  \thm@preskip=8pt plus 2pt minus 4pt
  \thm@postskip=\thm@preskip
}
\makeatother
\setlength{\parindent}{4em}
\setlength{\parskip}{0em}

\title{Engaging in Data Practices in Summer STEM Programs: A Person-in-Context Approach
}
\author{Joshua M. Rosenberg}
\fieldofstudy{Educational Psychology and Educational Technology}
\dedication{This dissertation is dedicated to Katie and to Jonah, who (mostly) happily slept through most of its writing.}
\date{2018}

%%%%%% MSU-THESIS stuff
% \usepackage[T1]{fontenc}
% \usepackage{newtxtext,newtxmath} % If they want Times we’ll give them Times
% \usepackage{amsmath}
% %
% \usepackage[]{natbib}
% \bibliographystyle{unified}
%
%
% % If you need newlines in your title, you must use \protect\\
% \title{Examining Work With Data in STEM Education Through the Lens of Engagement Theory: A Person-Oriented Approach Using an Experience Sampling Method}
% \author{Joshua M. Rosenberg}
% \fieldofstudy{Educational Psychology and Educational Technology}
% \dedication{This dissertation is dedicated to Katie.}
% \date{2018}
% \usepackage{listings}
% \lstset{language=TeX,basicstyle={\ttfamily}}
% \usepackage{lipsum}
% \usepackage{xcolor}
% \usepackage{gb4e}
%
% %\usepackage[bookmarksopenlevel=2,bookmarks=true]{hyperref} % not needed but here for testing
%
% \counterwithin{exx}{chapter}
% \singlegloss
%
% % Uncomment the next line for single spaced examples with gb4e
% %\patchcommand{\exe}{\SingleSpacing}{}
%
% % This code is an example of how to set up a new list of
% \newlistof{listoflistings}{lol}{List of Listings}
% \newfloat[chapter]{listing}{lol}{Listing}
% \newlistentry{listing}{lol}{0}
% \renewcommand*{\cftlistingname}{Listing\space}
% \renewcommand*{\cftlistingaftersnum}{\msucaptiondelim}
\usepackage{booktabs}
\usepackage{longtable}
\usepackage{array}
\usepackage{multirow}
\usepackage[table]{xcolor}
\usepackage{wrapfig}
\usepackage{float}
\usepackage{colortbl}
\usepackage{pdflscape}
\usepackage{tabu}
\usepackage{threeparttable}

\usepackage{amsthm}
\newtheorem{theorem}{Theorem}[section]
\newtheorem{lemma}{Lemma}[section]
\theoremstyle{definition}
\newtheorem{definition}{Definition}[section]
\newtheorem{corollary}{Corollary}[section]
\newtheorem{proposition}{Proposition}[section]
\theoremstyle{definition}
\newtheorem{example}{Example}[section]
\theoremstyle{definition}
\newtheorem{exercise}{Exercise}[section]
\theoremstyle{remark}
\newtheorem*{remark}{Remark}
\newtheorem*{solution}{Solution}
\begin{document}
%\maketitle

\maketitlepage
% Next make the abstract
\begin{abstract}
% Your abstract goes here.  Master's 1 page max. PhD 2 page max.
Data-rich activities provide an opportunity for science and mathematics learners to develop empowering capabilities. Aspects of work with data are recognized as core competencies in both science and mathematics curricular standards and have been the focus of research over the past three decades. While research on work with data has focused on cognitive outcomes and the development of specific practices at the student and classroom levels, little research has considered learners' experience--their perceptions of themselves, the activity, and of how engaged they are--of work with data and engaging in data science.

The present study explores learners engagement in data practices in the context of summer STEM programs. The data practices that are the focus of this study are selected on the basis of past research in science and mathematics education and data science education research. They are 1) asking questions, 2) observing phenomena, 3) constructing measures and generating data, 4) data modeling, and 5) interpreting findings. Because of the need to study learners' engagement in specific data practices, a person-in-context approach is used. Data from measures of learners' engagement ia collected through an Experience Sampling Method (ESM) that involves asking learners at random intervals to answer short questions about their experience and are analyzed with a person-oriented approach to discover profiles of learners' engagement. The following research questions guide the study: 1) What profiles of engagement and its conditions (PECs) emerge from the participants’ responses? 2) How does work with data relate to each PEC? 3) Do the relationships between instructional support for work with data and the PECs vary depending on students’ pre-program interest in STEM? 4) What are the common characteristics of potentially adaptive PECs beyond the presence of the aspects of work with data and other activities or the characteristics of learners?

These questions are explored in the context of nine summer STEM programs that took place over four week in one of two large cities in the Northeastern United States. 203 learners reported 2,970 responses via short ESM surveys of their perceptions of themselves (their competence) and of the activity (its challenge) and of how engaged they are. Programs were video-recorded, and segments of video associated with ESM responses were qualitatively coded for each of the data practices. Relations of learners engagement to the data practices were analyzed using multi-level models. Finally, activities were coded qualitatively to identify characteristics of particularly engaging activities

Aspects of work with data were fairly common overall, though modeling data was less common than other data practices. Relations of specific practices show that generating data is associated with particularly adaptive profiles (characterized by high levels of engagement and learners' positive perceptions of themselves and the activity), potentially because this step makes the work with data concrete to learners. This study provides an understanding of learners' experience of work with data and how work with data differs from other activities in summer STEM programs. Findings have implications for supporting work with data in informal and formal learning environments and for how researchers can use a person-in-context approach to study engaging in data science in a way that is sensitive to moment-to-moment changes in learners' experience.

\end{abstract}

% Force a newpage
\clearpage
% Make the copyright page. The Graduate School ridiculously prohibits you
% from having a copyright page unless you pay ProQuest to register the copyright.
% This should be illegal, but I didn't make up the rule.

\makecopyrightpage

% If you have a dedication page, uncomment the next command to print the dedication page
%
\makededicationpage
%
\clearpage

% Your Acknowledgements are formatted like a chapter, but with no number
\chapter*{Acknowledgements}
\DoubleSpacing % Acknowledgements should be double spaced
I would like to acknowledge my advisor and dissertation co-director Matthew Koehler and my dissertation co-director Jennifer Schmidt. I would also like to thank Lisa Linnenbrink-Garcia and Christina Schwarz. Thank you to my mentors and peers in the EPET program at MSU. Thank you to collaborators Lee Shumow and Neil Naftzger for their work on the STEM Interest and Engagement project (National Science Foundation DRL-1421198).Thank you to participating youth activity leaders and youth.
\clearpage

% We need to turn single spacing back on for the contents/figures/tables lists
\SingleSpacing
\tableofcontents* % table of contents will not be listed in the TOC
\clearpage
\listoftables % comment this out if you have no tables
\clearpage
\listoffigures % comment this out if you have no figures
\mainmatter
% If you have a list of abbreviations/symbols it would go here preceded by a \clearpage
% See the class documentation and the Memoir manual for how to create other lists
%

\chapter{Introduction}\label{intro}

\DoubleSpacing
