\documentclass[]{msu-thesis}
\usepackage{lmodern}
\usepackage{amssymb,amsmath}
\usepackage{ifxetex,ifluatex}
\usepackage{fixltx2e} % provides \textsubscript
\ifnum 0\ifxetex 1\fi\ifluatex 1\fi=0 % if pdftex
  \usepackage[T1]{fontenc}
  \usepackage[utf8]{inputenc}
\else % if luatex or xelatex
  \ifxetex
    \usepackage{mathspec}
  \else
    \usepackage{fontspec}
  \fi
  \defaultfontfeatures{Ligatures=TeX,Scale=MatchLowercase}
\fi
% use upquote if available, for straight quotes in verbatim environments
\IfFileExists{upquote.sty}{\usepackage{upquote}}{}
% use microtype if available
\IfFileExists{microtype.sty}{%
\usepackage{microtype}
\UseMicrotypeSet[protrusion]{basicmath} % disable protrusion for tt fonts
}{}
\usepackage[margin=1in]{geometry}
\usepackage{hyperref}
\hypersetup{unicode=true,
            pdftitle={Understanding Work With Data in Summer STEM Programs Through An Experience Sampling Method Approach},
            pdfauthor={Joshua M. Rosenberg},
            pdfborder={0 0 0},
            breaklinks=true}
\urlstyle{same}  % don't use monospace font for urls
\usepackage{natbib}
\bibliographystyle{apalike}
\usepackage{longtable,booktabs}
\usepackage{graphicx,grffile}
\makeatletter
\def\maxwidth{\ifdim\Gin@nat@width>\linewidth\linewidth\else\Gin@nat@width\fi}
\def\maxheight{\ifdim\Gin@nat@height>\textheight\textheight\else\Gin@nat@height\fi}
\makeatother
% Scale images if necessary, so that they will not overflow the page
% margins by default, and it is still possible to overwrite the defaults
% using explicit options in \includegraphics[width, height, ...]{}
\setkeys{Gin}{width=\maxwidth,height=\maxheight,keepaspectratio}
\IfFileExists{parskip.sty}{%
\usepackage{parskip}
}{% else
\setlength{\parindent}{0pt}
\setlength{\parskip}{6pt plus 2pt minus 1pt}
}
\setlength{\emergencystretch}{3em}  % prevent overfull lines
\providecommand{\tightlist}{%
  \setlength{\itemsep}{0pt}\setlength{\parskip}{0pt}}
\setcounter{secnumdepth}{5}
% Redefines (sub)paragraphs to behave more like sections
\ifx\paragraph\undefined\else
\let\oldparagraph\paragraph
\renewcommand{\paragraph}[1]{\oldparagraph{#1}\mbox{}}
\fi
\ifx\subparagraph\undefined\else
\let\oldsubparagraph\subparagraph
\renewcommand{\subparagraph}[1]{\oldsubparagraph{#1}\mbox{}}
\fi

%%% Use protect on footnotes to avoid problems with footnotes in titles
\let\rmarkdownfootnote\footnote%
\def\footnote{\protect\rmarkdownfootnote}

%%% Change title format to be more compact
\usepackage{titling}

% Create subtitle command for use in maketitle
\newcommand{\subtitle}[1]{
  \posttitle{
    \begin{center}\large#1\end{center}
    }
}

%\setlength{\droptitle}{-2em}
%  \title{Understanding Work With Data in Summer STEM Programs Through An Experience Sampling Method Approach}
%  \pretitle{\vspace{\droptitle}\centering\huge}
%  \posttitle{\par}
%  \author{Joshua M. Rosenberg}
%  \preauthor{\centering\large\emph}
%  \postauthor{\par}
%  \predate{\centering\large\emph}
%  \postdate{\par}
%  \date{2017-11-24}
%

\frontmatter
%\pagenumbering{Roman}
\newpage

\newpage

\pagebreak

\pagebreak


\usepackage{booktabs}
\usepackage{amsthm}
\makeatletter
\def\thm@space@setup{%
  \thm@preskip=8pt plus 2pt minus 4pt
  \thm@postskip=\thm@preskip
}
\makeatother
\setlength{\parindent}{4em}
\setlength{\parskip}{0em}

\title{Understanding Work With Data in Summer STEM Programs Through An Experience Sampling Method Approach}
\author{Joshua M. Rosenberg}
\fieldofstudy{Educational Psychology and Educational Technology}
\dedication{This dissertation is dedicated to Katie and to Jonah, who (mostly) happily slept through most of its writing.}
\date{2018}

%%%%%% MSU-THESIS stuff
% \usepackage[T1]{fontenc}
% \usepackage{newtxtext,newtxmath} % If they want Times we’ll give them Times
% \usepackage{amsmath}
% %
% \usepackage[]{natbib}
% \bibliographystyle{unified}
%
%
% % If you need newlines in your title, you must use \protect\\
% \title{Understanding Work With Data in Summer STEM Programs Through An Experience Sampling Method Approach}
% \author{Joshua M. Rosenberg}
% \fieldofstudy{Educational Psychology and Educational Technology}
% \dedication{This dissertation is dedicated to Katie and to Jonah, who (mostly) happily slept through most of its writing.}
% \date{2018}
% \usepackage{listings}
% \lstset{language=TeX,basicstyle={\ttfamily}}
% \usepackage{lipsum}
% \usepackage{xcolor}
% \usepackage{gb4e}
%
% %\usepackage[bookmarksopenlevel=2,bookmarks=true]{hyperref} % not needed but here for testing
%
% \counterwithin{exx}{chapter}
% \singlegloss
%
% % Uncomment the next line for single spaced examples with gb4e
% %\patchcommand{\exe}{\SingleSpacing}{}
%
% % This code is an example of how to set up a new list of
% \newlistof{listoflistings}{lol}{List of Listings}
% \newfloat[chapter]{listing}{lol}{Listing}
% \newlistentry{listing}{lol}{0}
% \renewcommand*{\cftlistingname}{Listing\space}
% \renewcommand*{\cftlistingaftersnum}{\msucaptiondelim}
\usepackage{booktabs}
\usepackage{longtable}
\usepackage{array}
\usepackage{multirow}
\usepackage[table]{xcolor}
\usepackage{wrapfig}
\usepackage{float}
\usepackage{colortbl}
\usepackage{pdflscape}
\usepackage{tabu}
\usepackage{threeparttable}

\usepackage{amsthm}
\newtheorem{theorem}{Theorem}[section]
\newtheorem{lemma}{Lemma}[section]
\theoremstyle{definition}
\newtheorem{definition}{Definition}[section]
\newtheorem{corollary}{Corollary}[section]
\newtheorem{proposition}{Proposition}[section]
\theoremstyle{definition}
\newtheorem{example}{Example}[section]
\theoremstyle{definition}
\newtheorem{exercise}{Exercise}[section]
\theoremstyle{remark}
\newtheorem*{remark}{Remark}
\newtheorem*{solution}{Solution}
\begin{document}
%\maketitle

\maketitlepage
% Next make the abstract
\begin{abstract}
% Your abstract goes here.  Master's 1 page max. PhD 2 page max.

Data-rich activities provide an opportunity for science and mathematics learners to develop empowering capabilities. Aspects of work with data are recognized have been core competencies in both science and mathematics curricular standards and have been the focus of research over the past three decades. While past research on work with data has focused on cognitive outcomes and the development of specific practices at the student and classroom levels, little research has considered learners' engagement and its conditions--including their perceptions of themselves and the activity--of work with data and engaging in data science. Contemporary engagement theory suggests the importance of considering youths' experiences and provides a framework for understanding and measuring them.

The present study explores learners engagement in work with data in the context of summer STEM programs. The aspects of work with data that are the focus of this study are selected on the basis of past research in science and mathematics education and data science education research. They are a) asking questions, b) observing phenomena, c) constructing measures and generating data, d) data modeling, and e) interpreting findings. Data from measures of learners' engagement is collected through the Experience Sampling Method (ESM) that involves asking learners at random intervals to answer short questions about their engagement and its conditions and are analyzed with a person-oriented approach to discover profiles of learners' engagement.

The following research questions guide the study: 1) What is the frequency and nature of opportunities for youth to engage in each of the five aspects of work with data in summer STEM programs? 2) What profiles of engagement emerge from data collected via ESM in the programs? 3) How do the five aspects of work with data relate to profiles of engagement? 4) How do youth characteristics relate to profiles of engagement?

These questions are explored in the context of nine summer STEM programs that took place over four week in large cities in the Northeastern United States. 203 learners reported 2,970 responses via short ESM surveys of how engaged they were (cognitively, behaviorally, and affectively, assessed through separate items) and of their perceptions of themselves (their competence) and of the activity (its challenge). Programs were video-recorded, and segments of video associated with ESM responses were qualitatively coded for each of the aspects of work with data. Relations of learners engagement to the aspects of work with data were analyzed using multi-level models. After being coded for work with data, activities were coded in an open-ended fashion to better understand the particular nature of work with data in this study's context.

Findings show that aspects of work with data were fairly common overall, though there were some misalignment revealed by the open-ended coding that suggests that some aspects were not carried out in a way that completely aligns with work with data as conceptualized and championed by past research. Six profiles of youth engagement that ranged from universally low to full, with intermediate profiles with d were identified with different configurations of cognitive, behavioral, and affective engagement, and youths' perceptions of competence and challenge. Relations of specific aspects of work with data show that generating data and data modeling are associated with full engagement, but that these effects were small; overall, work with data was not strongly related to the profiles. Female engaged in work with data as measured by a composite were more likely to be fully engaged, but, like for the relations of the individual aspects of work with data, this relationship was small, and relations overall were weak. Findings have implications for supporting work with data in informal and formal learning environments and for how researchers can use a person-in-context approach to study engaging in data science in a way that is sensitive to moment-to-moment changes in learners' experience.

\end{abstract}

% Force a newpage
\clearpage
% Make the copyright page. The Graduate School ridiculously prohibits you
% from having a copyright page unless you pay ProQuest to register the copyright.
% This should be illegal, but I didn't make up the rule.

\makecopyrightpage

% If you have a dedication page, uncomment the next command to print the dedication page
%
\makededicationpage
%
\clearpage

% Your Acknowledgements are formatted like a chapter, but with no number
\chapter*{Acknowledgements}
\DoubleSpacing % Acknowledgements should be double spaced
I would like to acknowledge my advisor and dissertation co-director Matthew Koehler and my dissertation co-director Jennifer Schmidt. I would also like to thank Lisa Linnenbrink-Garcia and Christina Schwarz. Thank you to my mentors and peers in the EPET program at MSU. Thank you to collaborators Lee Shumow and Neil Naftzger for their work on the STEM Interest and Engagement project (National Science Foundation DRL-1421198). Thank you to participating youth activity leaders and youth.
\clearpage

% We need to turn single spacing back on for the contents/figures/tables lists
\SingleSpacing
\tableofcontents* % table of contents will not be listed in the TOC
\clearpage
\listoftables % comment this out if you have no tables
\clearpage
\listoffigures % comment this out if you have no figures
\mainmatter
% If you have a list of abbreviations/symbols it would go here preceded by a \clearpage
% See the class documentation and the Memoir manual for how to create other lists
%

\chapter{Introduction}\label{intro}

\DoubleSpacing
